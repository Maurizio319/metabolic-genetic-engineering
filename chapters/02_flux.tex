\chapter{Flux}

\section{Introduction}
The metabolic flux can be defined as the rate at which material is processed through a metabolic pathway.
Along with intracellular metabolite concentrations, fluxes define the minimal information needed for describing metabolism and cell physiology.
Metabolic fluxes and their changes in response to various types of genetic and environmental perturbations are critical for the elucidation of the control of metabolic flux.

	\subsection{Feasibility and observability}
	A metabolic pathway is defined to be any sequence of feasible and observable biochemical reaction steps connecting a specified set of input and output metabolites.
	If the fluxes of reaction sequences cannot be determined independently, their inclusion provides no additional information.
	In many ways, it is preferable to lump thes reaction sequences together in fewer pathways whose overall fluxes can be experimentally observed.

\section{Metabolic flux analysis}
Metabolic flux analysis or MFA has been used over the past three decades to quantify intracellular metabolic fluxes in native and engineered biological systems.
Through MFA changes in metabolic pathway fluxes that result from genetic and or environmental interventions are quantified.
This information provides insights into the regulation of metabolic pathways and may suggest new targets for further metabolic engineering of the strains.

	\subsection{Flux analysis at metabolic steady-state: MFA}
	The first step in an MFA is to express the biochemical network model as a stoichiometric matrix in which rows represent balanced intracellular metabolites and columns represent metabolic fluxes in the model.
	The stoichiometric model includes a biomass reaction that describes the drain of precursor metabolites needed for cell growth, which is constructed based on the measured biomass composition.
	By assuming metabolic pseudo steady-state for intracellular metabolites, metabolic fluxes $\vec{v}$ are constrained by the stoichiometry matrix $S$:

	$$s\times\vec{v} = \vec{0}$$

	To estimate metabolic fluxes, the stoichiometric constraints are complemented with measured external metabolic rates, such as growth rate, substrate uptake and product accumulation rates, described by matrix $R$.
	This adds the constraint:

	$$R\times\vec{v}=\vec{r}$$

	The combined system of equation is solved by least squares regression:

	$$\min SSR = \sum\frac{(r-r_m)^2}{\sigma_r^2}$$

	With constraints $R\times\vec{v}=\vec{r}$ and $S\times v = \vec{0}$.
	MFA can estimate metabolic fluxes in systems fully or over-determined: all the necessary external rate measurement are known or they are redundant.
	This method is easy to apply and relies on relatively robust measurements of extracellular metabolites.
	However for many biological systems the number of constraints is often insufficient to observe all important metabolic pathways.
	To make the system fully observable additional assumptions are needed.
	For example specific pahthways that are assumed to carry little or no flux or cofactor balances are left out.

	\subsection{Flux balance analysis}
	Flux balance analysis or FBA can be applyed to quantify fluxes in underdetermined systems.
	In addition to applying constraints from measured extracellular rates, inequality constraints like upper and lower bounds on fluxes are used and an assumed biological objective is imposed on the model.
	FBA returns a large solution space consisting of many flux distributions that can all maximize the assumed cellular objective.

	\subsection{$^{13}$C-based metabolic flux analysis}
	$^{13}$C-based metabolic flux analysis os $^{13}$C-MFA is a more advanced technique for estimating metabolic fluxes in systems at metabolic steady state.
	$^{13}$C-labelled tracers, combined with isotopomer balancing, metabolite balancing and isotopic labelling measurement through techniques as NMR, mass spectrometry and tandem mass spectrometry are used to estimate fluxes.
	Cells are cultured for an extended period of time in the presence of a specifically labelled $^{13}$C-tracer which results in the incorporation of the tracer atoms into metabolic intermediates and products.
	The constraints

	$$f_{isotopomer}(\vec{x},\vec{v}) = \vec{0}$$

	Is introduced to account for isotopomer balancing.

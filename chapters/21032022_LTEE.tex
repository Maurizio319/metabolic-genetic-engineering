\

\section{Evolution}
The sequence of mutations matter for the evolution of mo, historical contingencies make evolution largely unpredictable: Although each change on an evolutionary path has some causal relation to the circumstances in which it arose, outcomes must eventually
depend on the details of long chains of antecedent states, small changes in which may have enormous long-term repercussions. Evolutionary preadaptation.

study evolution in real time. E coli, propagate them, try to understand what happens over time. LTEE Long-term evolution experiment is an experiment started in 1988 of \textit{Escherichia coli}, propagate 12 populations and something to distinguish them. They have since been propagated by daily 1:100 serial transfer in DM25, a minimal medium containing 25 mg/liter
glucose as the limiting resource.
Environmental conditions were mantained, now about 44000 generations. samples have been frozen every 500 generations  providing a rich "\textbf{\textit{fossil
record}}".

The founding strain is strictly asexual, and thus populations have evolved by natural selection and genetic drift acting on variation generated solely by spontaneous mutations that occurred during the experiment. Thus, the LTEE
allows the examination of the effects of contingency that are inherent to the core evolutionary processes of mutation, selection, and drift. A neutral change is generally happening, secondly lethal or disadvantageous. Genetic drift is when you pick randomly preferably a subset of the set of cells. 

All twelve populations underwent rapid improvement in fitness that
decelerated over time. All evolved higher maximum growth rates on glucose, shorter lag phases upon
transfer into fresh medium, reduced peak population densities, and larger
average cell sizes relative to their ancestor. Ten populations evolved increased DNA supercoiling. At least three genes have substitutions in all 12 populations , and several
others have substitutions in many populations.

At the same time, there has also been some divergence between populations.
Four have evolved defects in DNA repair, causing mutator phenotypes.
There is subtle, but significant, between population variation in mean fitness
in the glucose-limited medium in which they evolved.
In media containing other carbon sources, such as maltose or lactose, the
variation in performance is much greater. And while the same genes often
harbor substitutions, the precise location and details of the mutations almost
always differ between the populations.

Genetic sequencing

\begin{figure}[h]
\caption{Fluctuations, genetic changes not always important}
\centering
\includegraphics[width=0.6\textwidth]{}
\end{figure}

At a certain time, a flask from a single population generated turbidity, the growt of that bacteria was really high.

Citeria is used to control the pH of the culture. The inability to use citrate as an energy source
under oxic conditions has long been a defining characteristic of E. coli as a species. Although, E coli has all the genes necessary to metabolize citrate

E coli able to metabolize citrate have been isolated from agricultural and
clinical settings, and were found to harbor
plasmids, presumably acquired from other
species, that encode citrate transporters, so because of plasmids or other phenomena.

Just spontaneously E. coli was able to metabolize citrate, because of spontaneous mutations. 

\textbf{\textit{About the genetics}} of this one population \dots

The Cit+ trait originated in one clade by a tandem duplication that captured an aerobically expressed promoter for the expression of a previously silent citrate transporter. It set up the stage for a series of other mutations. Trying with the other ancestors, whose developed also the citrate character. The insertion site was different.  In these experiments, they observed 19 new, independent instances of
Cit+ re-evolution, but only when starting from clones isolated from after generation 20,000 (only from this generation).

\begin{figure}[h]
\caption{gltA1 was almost disappeared before the mutation. The one mutation somehow interacts with NADH, important ofr hte action of the enzyme.}
\centering
\includegraphics[width=0.6\textwidth]{}
\end{figure}


\subsection{LTEE: flux balance analysis (FBA)}

3 possible carbon sources: glucose, acetato, citrate. Acetate is a product of E. coli. Eat waste to stay alive. Both the evolved iclR and arcB alleles are loss-of- function mutations that
derepress expression of enzymes needed for acetate assimilation


if without gltA1, the rlativefitness decreases. gltA1 is involved in the interaction with NADH. Other changes seem to be important. The order of mutations matters. Cit++ is the best.

The other gltA2 mutation generates problems if 

3 stages neeeded to reach the citrate phase.
